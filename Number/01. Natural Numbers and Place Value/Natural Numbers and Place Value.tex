\documentclass[11pt, oneside]{article}

\usepackage{graphicx}
\usepackage{amssymb}
\usepackage{multirow}
\usepackage{float}
\usepackage{amsthm}
\usepackage[left=2cm, right=2cm, top=2cm]{geometry}
\usepackage{array}
\usepackage{pstricks-add}


\theoremstyle{definition}
\newtheorem{exmp}{Example}[section]

\begin{document}

\section{Factors of Natural Numbers}

Natural Numbers have many interesting properties that make them important. We know how to carry out operations with them such as addition, subtraction, multiplication and division. Some of these properties come from the results of carrying out such operations between them, in particular, division. 

When we divide two natural numbers, it is important to identify the different components in the calculation. For example, when we divide 47 by 3 using long division the calculation would be:


\psset{xunit=1.0cm,yunit=1.0cm,algebraic=true,dimen=middle,dotstyle=o,dotsize=5pt 0,linewidth=1.6pt,arrowsize=3pt 2,arrowinset=0.25}
\begin{pspicture}(-6.34,-2.28)(6.06,4.16)
\psline[linewidth=2.pt](-1.,2.)(-1.,3.)
\psline[linewidth=2.pt](-1.,3.)(1.42,2.96)
\rput[tl](-0.66,2.5){4}
\rput[tl](0.26,2.52){7}
\rput[tl](-1.78,2.46){3}
\rput[tl](-0.66,3.48){1}
\rput[tl](-0.66,1.52){3}
\psline[linewidth=2.pt](-1.,1.)(0.,1.)
\rput[tl](-0.72,0.42){1}
\rput[tl](0.3,0.46){7}
\rput[tl](0.24,3.48){5}
\rput[tl](-0.7,-0.48){1}
\rput[tl](0.24,-0.5){5}
\psline[linewidth=2.pt](-1.,-1.)(1.,-1.)
\rput[tl](0.28,-1.52){2}
\psline[linewidth=2.pt]{->}(0.46,1.94)(0.48,0.92)
\rput[tl](-5,2.6){divisor}
\psline[linewidth=2.pt]{->}(-3.72,2.46)(-1.98,2.46)
\rput[tl](3.1,2.6){dividend}
\psline[linewidth=2.pt]{->}(3.02,2.42)(0.8,2.42)
\rput[tl](3.1,3.65){quotient}
\psline[linewidth=2.pt]{->}(3.,3.42)(0.86,3.44)
\rput[tl](-4.5,-1.46){remainder}
\psline[linewidth=2.pt]{->}(-2.56,-1.6)(0.06,-1.6)
\end{pspicture}

The four key components in the division are the divisor, the dividend, the quotient and the remainder. When the remainder of a division is 0 we say that the divisor is a factor of the dividend. In the example of 47 divided by 3 we have a remainder of 2 so 3 is not a factor of 47. However, if we divide 45 by 3 we get a remainder of 0, so 3 is a factor of 45:

\psset{xunit=1.0cm,yunit=1.0cm,algebraic=true,dimen=middle,dotstyle=o,dotsize=5pt 0,linewidth=1.6pt,arrowsize=3pt 2,arrowinset=0.25}
\begin{pspicture*}(-6.34,-2.28)(6.06,4.16)
\psline[linewidth=2.pt](-1.,2.)(-1.,3.)
\psline[linewidth=2.pt](-1.,3.)(1.42,2.96)
\rput[tl](-0.66,2.5){4}
\rput[tl](0.26,2.52){5}
\rput[tl](-1.78,2.46){3}
\rput[tl](-0.66,3.48){1}
\rput[tl](-0.66,1.52){3}
\psline[linewidth=2.pt](-1.,1.)(0.,1.)
\rput[tl](-0.72,0.42){1}
\rput[tl](0.3,0.46){5}
\rput[tl](0.24,3.48){5}
\rput[tl](-0.7,-0.48){1}
\rput[tl](0.24,-0.5){5}
\psline[linewidth=2.pt](-1.,-1.)(1.,-1.)
\rput[tl](0.28,-1.52){0}
\psline[linewidth=2.pt]{->}(0.46,1.94)(0.48,0.92)
\rput[tl](-5,2.6){divisor}
\psline[linewidth=2.pt]{->}(-3.72,2.46)(-1.98,2.46)
\rput[tl](3.1,2.6){dividend}
\psline[linewidth=2.pt]{->}(3.02,2.42)(0.8,2.42)
\rput[tl](3.1,3.65){quotient}
\psline[linewidth=2.pt]{->}(3.,3.42)(0.86,3.44)
\rput[tl](-4.5,-1.46){remainder}
\psline[linewidth=2.pt]{->}(-2.56,-1.6)(0.06,-1.6)
\end{pspicture*}

Let's now look at some examples:

\begin{exmp} \end{exmp}
When we divide 34 by 5 we get a quotient of 6 and a remainder of 4, so 5 is not a factor of 35. However, when we divide 34 by 2 we get a quotient of 17 and a remainder of 0, so 2 is a factor of 34. 

\begin{exmp} \end{exmp}
When we divide 60 by 2 we get a remainder of 0, so 2 is a factor of 60. Similarly, when we divide 60 by 3 we also get a remainder of 0, so 3 is also a factor of 60. We can see that a Natural Number, in this case 60, can have many factors.

\begin{exmp} \end{exmp}
Let's try to find all the factors of 12. To do so we will try to divide 12 by all the Natural Numbers from 1 to 12 and check if the remainder is 0.
\begin{itemize}
\item 12 divided by 1 leaves a reminder of 0, so 1 is a factor of 12.
\item 12 divided by 2 leaves a reminder of 0, so 2 is a factor of 12.
\item 12 divided by 3 leaves a reminder of 0, so 3 is a factor of 12.
\item 12 divided by 4 leaves a reminder of 0, so 4 is a factor of 12.
\item 12 divided by 5 leaves a reminder of 2, so 5 is not a factor of 12.
\item 12 divided by 6 leaves a reminder of 0, so 6 is a factor of 12.
\item 12 divided by 7 leaves a reminder of 5, so 7 is not a factor of 12.
\item 12 divided by 8 leaves a reminder of 4, so 8 is not a factor of 12.
\item 12 divided by 9 leaves a reminder of 3, so 9 is not a factor of 12.
\item 12 divided by 10 leaves a reminder of 2, so 10 is not a factor of 12.
\item 12 divided by 11 leaves a reminder of 1, so 11 is not a factor of 12.
\item 12 divided by 12 leaves a reminder of 0, so 12 is a factor of 12.
\end{itemize}
Therefore, the factors of 12 are 1, 2, 3, 4, 6 and 12.

\bigbreak

\noindent\fbox{
    \parbox{\textwidth}{
{\bf A Different View of Factors:} We can also think of factors using multiplication. For example, if we want to check if 5 is a factor of 45, we simply check if there is a Natural Number which we can multiply 5 by to get 45 as an answer. In this case, $5\times 9 = 45$, so we can conclude that 5 is a factor of 45. However, there is no Natural Number that we can multiply by 6 to get 45 so 6 is not a factor of 45.
    }
}

\begin{exmp} \end{exmp}
Let's try to find all the factors of 10 using the alternative method: 
\begin{itemize}
\item We can multiply 1 by 10 to get 10, so 1 is a factor of 10.
\item We can multiply 2 by 5 to get 10, so 2 is a factor of 10.
\item We can multiply 5 by 2 to get 10, so 5 is a factor of 10.
\item We can multiply 10 by 1 to get 10, so 10 is a factor of 10.
\end{itemize}
For all the remaining natural numbers from 1 to 10 this is not possible so the factors of 10 are 1, 2, 5 and 10. We can also notice that the first case in this example is very similar to the last case, and the second case is very similar to the third. We could simply consider the example as follows:

\begin{itemize}
\item $10 = 1\times 10$, so 1 and 10 are factors of 10.
\item $10 = 2\times 5$, so 2  and 5 are factors of 10.
\end{itemize}
Since 10 can't be written as a product of two natural numbers in any other way, 1, 2, 5 and 10 are the factors of 10.

\bigbreak

\noindent\fbox{
    \parbox{\textwidth}{
{\bf An important conclusion:} when looking at the factors of a number, that number itself and 1 are always factors. For example, if we are looking at the factors of 49, we know that $49 = 1\times 49$ and so 1 and 49 are factors of 49.
    }
}


\subsection{Exercises}
\begin{enumerate}
\item Is 7 a factor of 156? Why or why not?
\item Is 7 a factor of 154? Why or why not?
\item Find all the factors of 15
\item Find all the factors of 16
\item Find all the factors of 17
\end{enumerate}



\end{document}  




















