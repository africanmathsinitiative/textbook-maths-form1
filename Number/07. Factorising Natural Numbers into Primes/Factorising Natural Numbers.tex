\documentclass[11pt, oneside]{article}

\usepackage{graphicx}
\usepackage{amssymb}
\usepackage{multirow}
\usepackage{float}
\usepackage{amsthm}
\usepackage[left=2cm, right=2cm, top=2cm]{geometry}
\usepackage{array}
\usepackage{pstricks-add}


\theoremstyle{definition}
\newtheorem{exmp}{Example}[section]

\begin{document}

\section{Factorising Natural Numbers into a product of primes}

Every Natural Number can be the result of a product, i.e. multiplication, of two or more natural numbers, some can be the result of many such multiplications. However, every single number can be the result of only a combination of prime numbers, for example $6 = 6\times 1 = 2 \times 3$. Here, only the second multiplication is a product of prime numbers. 

Similarly, $12 = 12 \times 1 = 6 \times 2 = 2\times 2 \times 3$. There are many ways to write 12 as a multiplication of natural numbers but if we are considering prime numbers $12 = 2\times 2\times 3$ is the only way in which this can be done. This process can be considered a way of breaking up numbers into a product of their most minimal components and is called factorising numbers into a product of primes. 

For small numbers, factorising into a product of primes can be simple and done by trying different possibilities, but realising that, for example, $324 = 2\times 2\times 3\times 3\times 3\times 3$ can be more complicated and for larger numbers even more time consuming. Fortunately, there is a process that will help you factorise any number, however, it is important to be familiar with prime numbers.

The process is simple, we start dividing our number by small prime numbers until we can no longer divide by them, then we move on to the next prime number, until we get 1 as a result of this division. Let's look at some examples.

\begin{exmp} \end{exmp}
\begin{tabular}{ p{0.5cm} | p{1.5cm}  p{10cm}}
36 & 2 & We start by dividing by the smallest prime, i.e. 2 \\
18 & 2 & We can still divide by 2  \\
9 & 3 & We can no longer divide by 2 so we divide by 3 \\
3 & 3 & We can still divide by 3 \\
1 &  & We reached 1 so the process finished.
\end{tabular}

We can now conclude that $36 = 2\times 2\times 3\times 3$. This is the product of the prime numbers on the right of the line.

Note that each time we divide by a prime number we place the result underneath in order to keep diving. In this example, $36 \div 2 = 18$ so we place the number 18 underneath the number 36. Similarly, $18 \div 2 = 9$ so we place the number 9 underneath the number 18.

\begin{exmp} \end{exmp}
\begin{tabular}{ p{0.5cm} | p{1.5cm}  p{10cm}}
54 & 2 & We start by dividing by 2 \\
27 & 3 & We can no longer divide by 2 so we divide by 3 \\
9 & 3 & We can still divide by 3 \\
3 & 3 & We can still divide by 3 \\
1 &  & We reached 1 so the process finished.
\end{tabular}

We can now conclude that $54 = 2\times 3\times 3\times 3$. This is the product of the prime numbers on the right of the line.

\begin{exmp} \end{exmp}
\begin{tabular}{ p{0.5cm} | p{1.5cm}  p{10cm}}
135 & 3 & We can't divide by 2 so we start by dividing by 3 \\
45 & 3 & We can still divide by 3  \\
15 & 3 & We can divide by 3 once again \\
5 & 5 & We now can only divide by 5 \\
1 &  & We reached 1 so the process finished.
\end{tabular}

We can now conclude that $135 = 3\times 3\times 3\times 5$. This is the product of the prime numbers on the right of the line.

\begin{exmp} \end{exmp}
\begin{tabular}{ p{0.5cm} | p{1.5cm}  p{10cm}}
490 & 2 & We start by dividing by 2 \\
245 & 5 & We can't divide by 2 or by 3 so we divide by 5  \\
49 & 7 & We can't divide by 5 so we divide by 7 \\
7 & 7 & We divide by 7 again \\
1 &  & We reached 1 so the process finished.
\end{tabular}

We can now conclude that $490 = 2\times 5\times 7\times 7$. This is the product of the prime numbers on the right of the line.

\begin{exmp} \end{exmp}
\begin{tabular}{ p{0.5cm} | p{1.5cm}  p{10cm}}
352 & 2 & We divide by 2 first \\
176 & 2 & We can still divide by 2  \\
88 & 2 & We can still divide by 2 \\
44 & 2 & We can still divide by 2 \\
22 & 2 & We can still divide by 2 \\
11 & 11 & We can't divide by 2, 3, 5 or 7, so we divide by 11 \\
1 &  & We reached 1 so the process finished.
\end{tabular}

We can now conclude that $352 = 2\times 2\times 2\times 2\times 2\times 11$. This is the product of the prime numbers on the right of the line.

\begin{exmp} \end{exmp}
\begin{tabular}{ p{0.5cm} | p{1.5cm}  p{14cm}}
637 & 7 & We check primes 2, 3 and 5 and see that we can't divide by them, so we divide by 7 \\
91 & 7 & We check with 7 and see that it still works \\
13 & 13 & We reached 13, which is a prime, so we divide by 13 \\
1 &  & We reached 1 so the process finished.
\end{tabular}

We can now conclude that $637 = 7\times 7\times 13$. This is the product of the prime numbers on the right of the line.

\subsection{Exercises}
Factorise the following numbers into primes:
\begin{enumerate}
\item $20 = $
\item $90 = $
\item $45 =$
\item $98 =$
\item $91 =$
\item $136 =$
\item $286 =$
\item $729 =$
\end{enumerate}



\end{document}  




















