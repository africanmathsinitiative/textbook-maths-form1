\documentclass[11pt, oneside]{article}

\usepackage{graphicx}
\usepackage{amssymb}
\usepackage{multirow}
\usepackage{float}
\usepackage{amsthm}
\usepackage[left=2cm, right=2cm, top=2cm]{geometry}
\usepackage{array}
\usepackage{pstricks-add}


\theoremstyle{definition}
\newtheorem{exmp}{Example}[section]

\begin{document}

\section{Highest Common Factor}

Imagine a textile worker who has two pieces of cloth of the same width, one 72 cm long and the second 64 cm long. She wants to cut the cloths into strips of equal length without wasting any materials. She will make ribbons with these strips and the best design uses strips as long as possible. What length should her strips be? 

Here, we can think of the possible sizes in which she can cut each cloth and compare them. As we don't want to have any waste, we need to find all the factors of 72 and 64. Let's start with 72:

\begin{itemize}
\item $72 = 1\times 72$
\item $72 = 2 \times 36$
\item $72 = 3 \times 24$
\item $72 = 4 \times 18$
\item $72 = 6\times 12$
\item $72 = 8\times 9$
\end{itemize} 
So we could cut the first cloth in pieces of length 1, 2, 3, 4, 6, 8, 9, 12, 18, 24, 36 or 72 cm.

The factors of 64 are:
\begin{itemize}
\item $64 = 1\times 64$
\item $64 = 2 \times 32$
\item $64 = 4 \times 16$
\item $64 = 8\times 8$
\end{itemize} 
So we could cut the second cloth in pieces of 1, 2, 4, 8, 16, 32 or 64 cm long.

Comparing these results we can see that the longest cuts could be 8 cm long. This is called the Highest Common Factor of 72 and 64. It is the largest number that is a factor of both 72 and 64. However, this method of finding the Highest Common Factor is not very efficient so we will look at a more efficient way of finding it using factorisation of Natural Numbers into primes:

\begin{tabular}{ p{0.5cm} | p{1.5cm}}
72 & 2  \\
36 & 2  \\
18 & 2  \\
9 & 3  \\
3 & 3 \\
1 &  
\end{tabular}

So $72 = 2 \times 2 \times 2 \times 3 \times 3$

\bigbreak

\begin{tabular}{ p{0.5cm} | p{1.5cm}}
64 & 2  \\
32 & 2  \\
16 & 2  \\
8 & 2  \\
4 & 2 \\
2 & 2 \\
1 
\end{tabular}

So $64 = 2 \times 2 \times 2 \times 2 \times 2 \times 2$

We now look at all the common prime factors and multiply them to obtain the highest common factor:

\begin{tabular}{c c c c c c c c c c c c c}
72 & = & {\bf2} & $\times $ & {\bf2} & $\times$  & {\bf2} & $\times$ & 3 & $\times $ & 3  &  &  \\
64 & = & {\bf2} & $\times $ & {\bf2} & $\times$  & {\bf2} & $\times$ & 2 & $\times $ & 2 & $\times $ & 2
\end{tabular}


In this case it would be $2\times 2\times 2 = 8$.

Let's now look at some examples:

\begin{exmp} \end{exmp}
We will find the Highest Common Factor of 30 and 50:

\begin{tabular}{ p{0.5cm} | p{1.5cm}}
30 & 2  \\
15 & 3  \\
5 & 5  \\
1 &  
\end{tabular}

So $30 = 2 \times 3 \times 5$

\bigbreak

\begin{tabular}{ p{0.5cm} | p{1.5cm}}
50 & 2  \\
25 & 5  \\
5 & 5  \\
1 
\end{tabular}

So $50 = 2 \times 5 \times 5$

We now look at all the common prime factors and multiply them to obtain the highest common factor:

\begin{tabular}{c c c c c c c}
30 & = & {\bf2} & $\times $ & 3 & $\times$  & {\bf5}  \\
50 & = & {\bf2} & $\times $ & {\bf5} & $\times$  & 5 
\end{tabular}

In this case it would be $2\times 5 = 10$.

\begin{exmp} \end{exmp}
We will find the Highest Common Factor of 90 and 63:

\begin{tabular}{ p{0.5cm} | p{1.5cm}}
90 & 2  \\
45 & 3  \\
15 & 3  \\
5 & 5  \\
1 &  
\end{tabular}

So $90 = 2 \times 3 \times 3 \times 5$

\bigbreak

\begin{tabular}{ p{0.5cm} | p{1.5cm}}
63 & 3  \\
21 & 3  \\
7 & 7  \\
1 & 
\end{tabular}

So $63 = 3 \times 3 \times 7$

We now look at all the common prime factors and multiply them to obtain the highest common factor:

\begin{tabular}{c c c c c c c c c}
90 & = & 2 & $\times $ & {\bf3} & $\times$  & {\bf3} & $\times$ & 5  \\
63 & = & {\bf3} & $\times $ & {\bf3} & $\times$  & 7 & &
\end{tabular}

In this case it would be $3\times 3 = 9$.


\begin{exmp} \end{exmp}
We will find the Highest Common Factor of 60 and 280:

\begin{tabular}{ p{0.5cm} | p{1.5cm}}
60 & 2  \\
30 & 2  \\
15 & 3  \\
5 & 5  \\
1 &  
\end{tabular}

So $60 = 2 \times 2 \times 3 \times 5$

\bigbreak

\begin{tabular}{ p{0.5cm} | p{1.5cm}}
280 & 2  \\
140 & 2  \\
70 & 2  \\
35 & 5  \\
7 & 7 \\
1 
\end{tabular}

So $280 = 2 \times 2 \times 2 \times 5 \times 7$

We now look at all the common prime factors and multiply them to obtain the highest common factor:

\begin{tabular}{c c c c c c c c c c c}
60 & = & {\bf2} & $\times $ & {\bf2} & $\times$  & 3 & $\times$ & {\bf5} &  &  \\
280 & = & {\bf2} & $\times $ & {\bf2} & $\times$  & 2 & $\times$ & {\bf5} & $\times $ & 7
\end{tabular}

In this case it would be $2\times 2\times 5 = 20$.

\begin{exmp} \end{exmp}
We will find the Highest Common Factor of 75 and 56:

\begin{tabular}{ p{0.5cm} | p{1.5cm}}
75 & 3  \\
25 & 5  \\
5 & 5  \\
1 &  
\end{tabular}

So $75 = 3 \times 5 \times 5$

\bigbreak

\begin{tabular}{ p{0.5cm} | p{1.5cm}}
56 & 2  \\
28 & 2  \\
14 & 2  \\
7 & 7  \\
1 
\end{tabular}

So $56 = 2 \times 2 \times 2 \times 7$

We now look at all the common prime factors and multiply them to obtain the highest common factor:

\begin{tabular}{c c c c c c c c c}
75 & = & 3 & $\times $ & 5 & $\times$  & 5 & &  \\
56 & = & 2 & $\times $ & 2 & $\times$  & 2 & $\times$ & 7
\end{tabular}

In this case we haven't got any common prime factors. Since 1 is a factor of every number, the Highest Common Factor of 75 and 56 is 1.

\bigbreak

This process can also be done if we are trying to find the Highest Common Factor of more numbers. In the next two examples we will look at three and four numbers:

\begin{exmp} \end{exmp}
We will find the Highest Common Factor of 36, 28 and 42:

\begin{tabular}{ p{0.5cm} | p{1.5cm}}
36 & 2  \\
18 & 2  \\
9 & 3  \\
3 & 3  \\
1 &  
\end{tabular}

So $36 = 2 \times 2 \times 3 \times 3$

\bigbreak

\begin{tabular}{ p{0.5cm} | p{1.5cm}}
28 & 2  \\
14 & 2  \\
7 & 7  \\
1 
\end{tabular}

So $28 = 2 \times 2 \times 7$

\bigbreak

\begin{tabular}{ p{0.5cm} | p{1.5cm}}
42 & 2  \\
21 & 3  \\
7 & 7  \\
1 
\end{tabular}

So $42 = 2 \times 3 \times 7$

We now look at all the common prime factors and multiply them to obtain the highest common factor:

\begin{tabular}{c c c c c c c c c}
36 & = & {\bf2} & $\times $ & 2 & $\times$  & 3 & $\times$ & 3  \\
28 & = & {\bf2} & $\times $ & 2 & $\times$  & 7 &  & \\
42 & = & {\bf2} & $\times $ & 3 & $\times$  & 7 &  &
\end{tabular}

In this case the only prime factor common to all three numbers is 2, which is the Highest Common Factor of 36, 28 and 42.

\begin{exmp} \end{exmp}
We will find the Highest Common Factor of 84, 60, 66 and 210:

\begin{tabular}{ p{0.5cm} | p{1.5cm}}
84 & 2  \\
42 & 2  \\
21 & 3  \\
7 & 7  \\
1 &  
\end{tabular}

So $84 = 2 \times 2 \times 3 \times 7$

\bigbreak

\begin{tabular}{ p{0.5cm} | p{1.5cm}}
60 & 2  \\
30 & 2  \\
15 & 3  \\
5 & 5 \\
1 
\end{tabular}

So $60 = 2 \times 2 \times 3 \times 5$

\bigbreak

\begin{tabular}{ p{0.5cm} | p{1.5cm}}
66 & 2  \\
33 & 3  \\
11 & 11 \\
1 
\end{tabular}

So $66 = 2 \times 3 \times 11$

\bigbreak

\begin{tabular}{ p{0.5cm} | p{1.5cm}}
210 & 2  \\
105 & 3  \\
35 & 5 \\
7 & 7 \\
1 
\end{tabular}

So $210 = 2 \times 3 \times 5 \times 7$

We now look at all the common prime factors and multiply them to obtain the highest common factor:

\begin{tabular}{c c c c c c c c c}
84 & = & {\bf2} & $\times $ & 2 & $\times$  & {\bf3} & $\times$ & 7  \\
60 & = & {\bf2} & $\times $ & 2 & $\times$  & {\bf3} & $\times $ & 5 \\
66 & = & {\bf2} & $\times $ & {\bf3} & $\times$  & 11 &  & \\
210 & = & {\bf2} & $\times $ & {\bf3} & $\times$  & 5 & $\times $ & 7
\end{tabular}

In this case the prime factors common to all four numbers are 2 and 3 and the Highest Common Factor is $2\times 3 = 6$.


\subsection{Exercises}
Find the Highest Common Factor of the following sets of numbers:
\begin{enumerate}
\item 12 and 18
\item 36 and 64
\item 100 and 96
\item 264 and 198
\item 104, 260 and 650
\item 126, 189, 630 and 525
\bigbreak
\item An agricultural school has a rectangular field of 100 metres by 150 metres that needs to be split into square plots to distribute to groups of students for their practical project. Each group will need a square plot as large as possible and no land is to be left unused. What will the dimension of the square plots be? {\bf Challenge:} How many plots will there be?
\end{enumerate}

\end{document}  




















