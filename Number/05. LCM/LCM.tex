\documentclass[11pt, oneside]{article}

\usepackage{graphicx}
\usepackage{amssymb}
\usepackage{multirow}
\usepackage{float}
\usepackage{amsthm}
\usepackage[left=2cm, right=2cm, top=2cm]{geometry}
\usepackage{array}
\usepackage{pstricks-add}


\theoremstyle{definition}
\newtheorem{exmp}{Example}[section]

\begin{document}

\section{Lowest Common Multiple}

Two long distance runners go training at a running circuit and start together at the same time. The first one takes 10 minutes to complete the circuit and the second takes 12 minutes. How long do they have to run for if they want to stop training when they meet again at the starting point? To find out when they will meet again we can see when each will be at the starting point and notice they coincide. The first runner takes 10 minutes, se we need to consider the multiples of 10 to find every time she reaches the starting line. The first ten multiples of 10 are 10, 20, 30, 40, 50, 60, 70, 80, 90 and 100. Similarly, the second runner will cross the start line at multiples of 12 as she takes 12 minutes to complete the track. The first ten multiples of 12 are 12, 24, 36, 48, 60, 72, 84, 96, 108 and 120. Comparing the lists of multiples we can conclude that they will both be together at the starting line at the same time is 60 minutes after starting, so they will finish their training after 60 minutes. This number is the smallest number that is a multiple of both 10 and 12 and can be very useful in many contexts. This is called the Lowest Common Multiple.

This process can be quite cumbersome and might take longer than we expect, particularly for larger numbers. Fortunately, we can find Lowest Common Multiples of sets of numbers by studying their prime factorisation. Let's consider the prime factorisation of 10 and 12:

\begin{tabular}{ p{0.5cm} | p{1.5cm}}
10 & 2  \\
5 & 5  \\
1 &  
\end{tabular}

So $10 = 2 \times 5$

\bigbreak

\begin{tabular}{ p{0.5cm} | p{1.5cm}}
12 & 2  \\
6 & 2  \\
3 & 3  \\
1 
\end{tabular}

So $12 = 2 \times 2 \times 3$

We now look at all the prime factors in order to compare them:

\begin{tabular}{c c c c c c c c c c c c c}
10 & = & 2 & $\times$ & 5 &  &  &  &  &  &  &  &  \\
12 & = & 2 & $\times$ & 2 & $\times$ & 3 &  &  &  &  &  & 
\end{tabular}

All the multiples of 10 will have to have 2 and 5 as prime factors. All the multiples of 12 will have 2, 2 and 3 as prime factors. So to find the Lowest Common Multiple we need to include 2 two times, 3 once and 5 once to make sure that it is a multiple of both 10 and 12. This gives us a Lowest Common Multiple of $2\times 2 \times 3 \times 5 = 60$.

Let's now look at some examples:

\begin{exmp} \end{exmp}
To find the Lowest Common Multiple of 18 and 24:

\begin{tabular}{ p{0.5cm} | p{1.5cm}}
18 & 2  \\
9 & 3  \\
3 & 3  \\
1 &  
\end{tabular}

So $18 = 2 \times 3 \times 3$

\bigbreak

\begin{tabular}{ p{0.5cm} | p{1.5cm}}
24 & 2  \\
12 & 2  \\
6 & 2  \\
3 & 3  \\
1 
\end{tabular}

So $24 = 2\times 2 \times 2 \times 3$

We now look at all the prime factors in order to compare them:

\begin{tabular}{c c c c c c c c c c c c c}
18 & = & 2 & $\times$ & 3 & $\times$ & 3 &  &  &  &  &  &  \\
24 & = & 2 & $\times$ & 2 & $\times$ & 2 & $\times$ & 3 &  &  &  & 
\end{tabular}

We notice that 2 appears three times in the factorisation of 24, and two times in the factorisation of 18, se we need 2 to appear three times. Similarly, 3 appears twice in the factorisation of 18 and once in the factorisation of 24, so we need it to appear twice. So the Lowest Common Multiple of 18 and 24 is $2\times 2 \times 2 \times 3 \times 3 = 72$.

\begin{exmp} \end{exmp}
To find the Lowest Common Multiple of 48 and 20:

\begin{tabular}{ p{0.5cm} | p{1.5cm}}
48 & 2  \\
24 & 2  \\
12 & 2  \\
6 & 2  \\
3 & 3  \\
1 &  
\end{tabular}

So $48 = 2 \times 2 \times 2 \times 2 \times 3$

\bigbreak

\begin{tabular}{ p{0.5cm} | p{1.5cm}}
20 & 2  \\
10 & 2  \\
5 & 5  \\
1 
\end{tabular}

So $20 = 2\times 2 \times 5$

We now look at all the prime factors in order to compare them:

\begin{tabular}{c c c c c c c c c c c c c}
48 & = & 2 & $\times$ & 2 & $\times$ & 2 & $\times$ & 2 & $\times$ & 3 &  &  \\
20 & = & 2 & $\times$ & 2 & $\times$ & 5 &  &  &  &  &  & 
\end{tabular}

We notice that 2 appears four times in the factorisation of 48, 3 appears once in the factorisation of 48 and 5 appears once in the factorisation of 20, and this combination includes all the prime factors of both numbers. So the Lowest Common Multiple of 48 and 20 is $2\times 2 \times 2 \times 2 \times 3 \times 5 = 240$.

\begin{exmp} \end{exmp}
To find the Lowest Common Multiple of 84 and 198:

\begin{tabular}{ p{0.5cm} | p{1.5cm}}
84 & 2  \\
42 & 2  \\
21 & 3  \\
7 & 7  \\
1 &  
\end{tabular}

So $84 = 2 \times 2 \times 3 \times 7$

\bigbreak

\begin{tabular}{ p{0.5cm} | p{1.5cm}}
198 & 2  \\
99 & 3  \\
33 & 3  \\
11 & 11 \\
1 
\end{tabular}

So $198 = 2\times 3 \times 3 \times 11$

We now look at all the prime factors in order to compare them:

\begin{tabular}{c c c c c c c c c c c c c}
84 & = & 2 & $\times$ & 2 & $\times$ & 3 & $\times$ & 7 &  &  &  &  \\
198 & = & 2 & $\times$ & 3 & $\times$ & 3 & $\times $ & 11 &  &  &  & 
\end{tabular}

We notice that 2 appears twice in the factorisation of 84, 3 appears twice in the factorisation of 198, 7 appears once in the factorisation of 84 and 11 appears once in the factorisation of 198, and this combination includes all the prime factors of both numbers. So the Lowest Common Multiple of 84 and 198 is $2\times 2 \times 3 \times 3 \times 7 \times 11 = 2,772$.

\begin{exmp} \end{exmp}
To find the Lowest Common Multiple of 36 and 72:

\begin{tabular}{ p{0.5cm} | p{1.5cm}}
36 & 2  \\
18 & 2  \\
9 & 3  \\
3 & 3  \\
1 &  
\end{tabular}

So $36 = 2 \times 2 \times 3 \times 3$

\bigbreak

\begin{tabular}{ p{0.5cm} | p{1.5cm}}
72 & 2  \\
36 & 2  \\
18 & 2  \\
9 & 3  \\
3 & 3  \\
1 &  
\end{tabular}

So $72 = 2\times 2 \times 2 \times 3 \times 3$

We now look at all the prime factors in order to compare them:

\begin{tabular}{c c c c c c c c c c c c c}
36 & = & 2 & $\times$ & 2 & $\times$ & 3 & $\times$ & 3 &  &  &  &  \\
72 & = & 2 & $\times$ & 2 & $\times$ & 2 & $\times $ & 3 & $\times$ & 3 &  & 
\end{tabular}

We notice that 2 appears three times in the factorisation of 72 and 3 appears twice in the factorisation of 72, and this combination includes all the prime factors of both numbers. So the Lowest Common Multiple of 72 and 36 is $2\times 2 \times 2 \times 3 \times 2 = 72$. Notice that the Lowest Common Multiples of 36 and 72 is 72 and this is because it is a multiple of 36. If we are looking for the Lowest Common Multiple of two numbers and the second is a multiple of the first, the Lowest Common Multiple will always be the second, as it is in this example.

This process will be the same when we need to find the Lowest Common Multiple of more than two numbers, as the next two examples demonstrate:

\begin{exmp} \end{exmp}
To find the Lowest Common Multiple of 15, 20 and 50:

\begin{tabular}{ p{0.5cm} | p{1.5cm}}
15 & 3  \\
5 & 5  \\
1 &  
\end{tabular}

So $15 = 3 \times 5$

\bigbreak

\begin{tabular}{ p{0.5cm} | p{1.5cm}}
20 & 2  \\
10 & 2  \\
5 & 5  \\
1 
\end{tabular}

So $20 = 2\times 2 \times 5$

\bigbreak

\begin{tabular}{ p{0.5cm} | p{1.5cm}}
50 & 2  \\
25 & 5  \\
5 & 5  \\
1 
\end{tabular}

So $50 = 2\times 5 \times 5$

We now look at all the prime factors in order to compare them:

\begin{tabular}{c c c c c c c c c c c c c}
15 & = & 3 & $\times$ & 5 &  &  &  &  &  &  &  &  \\
20 & = & 2 & $\times$ & 2 & $\times$ & 5 &  &  &  &  &  & \\
50 & = & 2 & $\times$ & 5 & $\times$ & 5 &  &  &  &  &  & 
\end{tabular}

We notice that 2 appears twice in the factorisation of 20, 3 appears once in the factorisation of 20, and 5 appears twice in the factorisation of 50, and this combination includes all the prime factors of all three numbers. So the Lowest Common Multiple of 15, 20 and 50 is $2\times 2 \times 3 \times 5 \times 5 = 300$. 

\begin{exmp} \end{exmp}
To find the Lowest Common Multiple of 182, 210, 27 and 165:

\begin{tabular}{ p{0.5cm} | p{1.5cm}}
182 & 2  \\
91 & 7  \\
13 & 13 \\
1 
\end{tabular}

So $182 = 2 \times 7 \times 13$

\bigbreak

\begin{tabular}{ p{0.5cm} | p{1.5cm}}
210 & 2  \\
105 & 3  \\
35 & 5  \\
7 & 7 \\
1
\end{tabular}

So $210 = 2\times 3 \times 5 \times 7$

\bigbreak

\begin{tabular}{ p{0.5cm} | p{1.5cm}}
27 & 3  \\
9 & 3  \\
3 & 3  \\
1 
\end{tabular}

So $27 = 3\times 3 \times 3$

\bigbreak

\begin{tabular}{ p{0.5cm} | p{1.5cm}}
165 & 3  \\
55 & 5  \\
11 & 11 \\
1 
\end{tabular}

So $165 = 3\times 5 \times 11$

We now look at all the prime factors in order to compare them:

\begin{tabular}{c c c c c c c c c c c c c}
182 & = & 2 & $\times$ & 7 & $\times$ & 13 &  &  &  &  &  &  \\
210 & = & 2 & $\times$ & 3 & $\times$ & 5 & $\times$ & 7 &  &  &  & \\
27 & = & 3 & $\times$ & 3 & $\times$ & 3 &  &  &  &  &  & \\
165 & = & 3 & $\times$ & 5 & $\times$ & 11 &  &  &  &  &  & 
\end{tabular}

We notice that 2 appears once in the factorisation of 182 and 210, 3 appears three times in the factorisation of 27, 5 appears once in the factorisation of 210 and 165, 7 appears once in the factorisation of 182 and 210, 11 appears once in the factorisation of 165 and 13 appears once in the factorisation of 182, and this combination includes all the prime factors of all three numbers. So the Lowest Common Multiple of 182, 210, 27 and 165 is $2\times 3 \times 3 \times 3 \times 5 \times 7 \times 11 \times 13 = 270,270$. Note that when we are dealing with many numbers which don't have many common prime factors the Lowest Common Multiple will be a rather large number, as it is in this case.

\subsection{Exercises}
Find the Lowest Common Multiple of the following sets of numbers:
\begin{enumerate}
\item 15 and 24
\item 50 and 40
\item 56 and 54
\item 48 and 96
\item 120, 48 and 35
\item 28, 36, 60 and 88
\item 66, 117, 425 and 245
\bigbreak
\item Three kids go to a games fair and get tokens each time they play a game. The first kid chooses a game that gives 12 tokens, the second chooses one that gives 15 tokens and the third chooses one that gives 20 tokens. They decide to keep playing their chosen games until all three have the same number of tokens. How many tokens do they have to win before they stop playing? {\bf Challenge:} How many times will each kid play their respective game?
\end{enumerate}

\end{document}  




















